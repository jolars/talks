% Options for packages loaded elsewhere
\PassOptionsToPackage{unicode,pdfusetitle}{hyperref}
\PassOptionsToPackage{hyphens}{url}
\PassOptionsToPackage{dvipsnames,svgnames*,x11names*}{xcolor}

\documentclass[10pt]{beamer}
\usepackage{lmodern}
\usepackage{amssymb,amsmath,mathtools,amsthm}
\usepackage[T1]{fontenc}
\usepackage[utf8]{inputenc}
\usepackage{textcomp} % provide euro and other symbols

\usepackage{pgfpages}

% prevent slide breaks in the middle of a paragraph
\widowpenalties 1 10000
%\raggedbottom

% redefine part, section, and subsection headers
%\raggedbottom
\setbeamertemplate{part page}{
 \centering
 \begin{beamercolorbox}[sep=16pt,center]{part title}
    \usebeamerfont{part title}\insertpart\par
 \end{beamercolorbox}
}
\setbeamertemplate{section page}{
 \centering
 \begin{beamercolorbox}[sep=12pt,center]{part title}
    \usebeamerfont{section title}\insertsection\par
 \end{beamercolorbox}
}
\setbeamertemplate{subsection page}{
 \centering
 \begin{beamercolorbox}[sep=8pt,center]{part title}
    \usebeamerfont{subsection title}\insertsubsection\par
 \end{beamercolorbox}
}

\AtBeginPart{\frame{\partpage}}
\AtBeginSection{\ifbibliography\else\frame{\sectionpage}\fi}
\AtBeginSubsection{\frame{\subsectionpage}}

% beamer configuration
\usecolortheme{dove}
\usefonttheme{professionalfonts}
\usefonttheme{structurebold}
\setbeamertemplate{footline}[frame number]
\setbeamertemplate{caption}[numbered]
\setbeamertemplate{caption label separator}{: }
\setbeamercolor{caption name}{fg=normal text.fg}
\setbeamertemplate{frametitle}{\begin{centering}\insertframetitle\par\end{centering}}
\setbeamertemplate{itemize items}[circle]
\setbeamerfont{frametitle}{size=\large}
\setbeamertemplate{headline}{\vskip5ex}
\beamertemplatenavigationsymbolsempty
%\setlength{\parskip}{1em} % add paragraph spacing

% Use upquote if available, for straight quotes in verbatim environments
\usepackage{upquote}
\usepackage[]{microtype}
\UseMicrotypeSet[protrusion]{basicmath} % disable protrusion for tt fonts

\usepackage{xcolor}
\usepackage{xurl} % add URL line breaks if available
\usepackage{bookmark}
\usepackage{hyperref}
\hypersetup{
  colorlinks=true,
  linkcolor=Maroon,
  filecolor=Maroon,
  citecolor=Blue,
  urlcolor=Blue
}

\usepackage[inline]{asymptote}

\usepackage{tikz}
\usetikzlibrary{arrows,shapes,positioning,intersections}

\usepackage{pgfplots}
\usepgfplotslibrary{external,colormaps}
\pgfplotsset{width=7cm,compat=1.11}
%\tikzexternalize

%\usepackage{tikz}
%\usepackage{pgfplots}
%\usepackage[mode=buildnew]{standalone}% requires -shell-escape
%\usetikzlibrary{intersections}

\urlstyle{same} % disable monospaced font for URLs
\newif\ifbibliography
\setlength{\emergencystretch}{3em} % prevent overfull lines
\setcounter{secnumdepth}{-\maxdimen} % remove section numbering

%\usepackage{subfig}
\usepackage{subcaption}
\usepackage{algorithm,algpseudocode}
\usepackage{booktabs}
\DeclareMathOperator{\E}{\text{E}}
\DeclareMathOperator{\Var}{var}
\DeclareMathOperator{\Cov}{cov}

% biblatex
\usepackage[citestyle=authoryear]{biblatex}
\addbibresource{ml-group-slope.bib}

\title{SLOPE}
\subtitle{Presentation for the ML group at LUSEM}
\author{Johan Larsson}
\date{May 27, 2020}
\institute{Department of Statistics, Lund University}
\titlegraphic{\includegraphics{lu.pdf}}

\begin{document}
\frame[noframenumbering,plain]{\titlepage}

\begin{frame}{Overview}
\tableofcontents
\end{frame}

\section{Introducing SLOPE}

\begin{frame}{Motivation and setting}
\protect\hypertarget{sorted-l1}{}
\begin{block}{setting}
    want to apply a \alert{generalized linear model}\footnote{least-squares, logistic, multinomial, or Poisson regression for instance} to a set of predictors \(X \in \mathbb{R}^{n\times p}\) and outcome
    \(y \in \mathbb{R}^n\)\medskip
    
    leads to finding the optimal solution (\(\hat\beta\)) to the problem
    \[
        \text{minimize} \quad g(\beta;X,y),
    \]
    for instance \(g(\beta; X,y) \coloneqq \frac 12 \lVert y - X\beta \rVert_2^2\) for OLS.
\end{block}
\pause
\begin{block}{problem}
\begin{itemize}
    \item \(p \gg n\)
    \item believe \alert{real} \(\beta\) is sparse: few signals, much noise
    \item want to avoid overfitting
    \item want efficiency
\end{itemize}
\end{block}
\end{frame}

\begin{frame}{Regularization}
\begin{block}{idea}
introduce regularization by constraining the problem, i.e. solve
\[
    \begin{aligned}
        &\text{minimize}   && g(\beta;X,y)\\
        &\text{subject to} && h(\beta) \leq t
    \end{aligned}
\]
and choose \(h(\beta)\) such that the resulting model is \alert{sparse} by shrinking some 
elements in \(\beta\) to be \emph{exactly} zero.
\end{block}
\begin{block}{typical choices for \(h(\beta)\)}
\begin{description}
    \item[\(\ell_0\) norm] best subset selection
    \item[\(\ell_1\) norm] the lasso
\end{description}
\end{block}
\end{frame}

\begin{frame}{Problems with standard methods}
\begin{block}{best subset selection}
\begin{itemize}
    \item not convex and therefore intractable for large \(p\)
    \item no shrinkage
\end{itemize}
\end{block}
\begin{block}{lasso}
\begin{itemize}
    \item unpredictable model selection for highly correlated predictors
    \item can only select \(n\) coefficients
\end{itemize}
\end{block}
\end{frame}

\begin{frame}{SLOPE}
\textcite{bogdan2015} introduced SLOPE (Sorted L-One Penalized Estimation), which solves
the problem
\[
\begin{aligned}
    &\text{minimize}   && g(\beta; X,y) \\
    &\text{subject to} && \sum_{i=1}^n \lambda_i \lvert \beta \rvert_{(i)} \leq t,
\end{aligned}
\]
where \(\sum_{i=1}^p\lambda_i \lvert \beta \rvert_{(i)}\) is the \alert{sorted \(\ell_1\) norm},
for which
\[\lambda_1 \geq \lambda_2 \geq \cdots \geq \lambda_p \geq 0\]
and
\[\lvert \beta \rvert_{(1)} \geq \lvert \beta \rvert_{(2)} \geq \cdots \geq \lvert \beta \rvert_{(p)}.\]
\end{frame}

\begin{frame}[fragile]{A step back: least squares}

for simplicity, let's assume \[g(\beta;X,y) = \frac 12 \lVert X\beta - y\rVert_2^2,\] i.e., we 
are solving (unregularized) OLS---solution available analytically through
normal equations.

\begin{center}
\begin{asy}
    import graph;
    import contour;
    
    size(180, 180);
    
    pair x0 = (1.6, 0.8);
    
    dot(x0);
    draw(circle(x0, 0.822), dashed);
    draw(circle(x0, 1.3), dashed);
    draw(circle(x0, 0.344), dashed);
    
    pair p1 = (1.6, 0.8);
    dot("$\hat\beta$", p1);
    
    xaxis("$\beta_1$", xmin = -1.1);
    yaxis("$\beta_2$", ymin = -1.1);
\end{asy}
\end{center}
\end{frame}

\begin{frame}[fragile]{Constraint region for the sorted \(\ell_1\) norm}
\(\sum_{i=1}^p \lambda_i \lvert \beta \rvert_{(i)} \leq t\) defines a 
constraint region centered at \(\boldsymbol{0}\)
\begin{center}
\begin{asy}
    import graph;
    import contour;
    
    size(150, 150);
    
    path p = (-1,0)--(-0.75,0.75)--
             (0,1)--(0.75,0.75)--
             (1,0)--(0.75,-0.75)--
             (0,-1)--(-0.75,-0.75)--(-1,0);
    draw(p);
    
    xaxis("$\beta_1$", xmax = 1.2, xmin = -1.2);
    yaxis("$\beta_2$", ymin = -1.2, ymax = 1.2);
\end{asy}
\end{center}
\end{frame}

\begin{frame}[fragile]{Sorted \(\ell_1\)-regularized least squares}
    let's introduce regularization via the sorted \(\ell_1\) norm: solution \(\hat\beta\)
    has to lie inside constraint region defined by 
    \[\sum_{i=1}^p \lambda_i\lvert \beta \rvert_{(i)} \leq t.\]
    \begin{center}
        \begin{asy}
            import graph;
            import contour;
            
            size(180, 180);
            
            path p = (-1,0)--(-0.75,0.75)--
                     (0,1)--(0.75,0.75)--
                     (1,0)--(0.75,-0.75)--
                     (0,-1)--(-0.75,-0.75)--(-1,0);
            draw(p);
            draw(scale(0.5)*p);
            
            pair x0 = (1.6, 0.8);
            
            dot(x0);
            draw(circle(x0, 0.822), dashed);
            draw(circle(x0, 1.3), dashed);
            draw(circle(x0, 0.344), dashed);
            
            pair p1 = (0.75/2, 0.75/2);
            dot("$\beta^*_{\lambda^{(1)}}$", p1);
            
            pair p2 = (0.805, 0.59);
            dot("$\beta^*_{\lambda^{(2)}}$", p2);
            
            xaxis("$\beta_1$", xmin = -1.1);
            yaxis("$\beta_2$", ymin = -1.1);
        \end{asy}
    \end{center}
\end{frame}

\begin{frame}[fragile]{Shapes of the sorted \(\ell_1\) norm}
    choice of \(\lambda\) dictates shape of the constraint region
    \begin{figure}[hbtp]
        \begin{subfigure}[b]{.3\linewidth}
            \centering
            \begin{asy}
                import graph;
                import contour;
                
                size(80, 80);
                
                path p = (-1,0)--(0,1)--(1,0)--(0,-1)--(-1,0);
                draw(p);
                xaxis("$\beta_1$", xmax = 1.5, xmin = -1.1);
                yaxis("$\beta_2$", ymax = 1.5, ymin = -1.1);
            \end{asy}
            \caption{\(\lambda_1 = \lambda_2\)}
        \end{subfigure}
        \begin{subfigure}[b]{.3\linewidth}
            \begin{asy}
                import graph;
                import contour;
                
                size(80, 80);
                
                path p = (-1,0)--(-0.75,0.75)--
                         (0,1)--(0.75,0.75)--
                         (1,0)--(0.75,-0.75)--
                         (0,-1)--(-0.75,-0.75)--(-1,0);
                draw(p);
                xaxis("$\beta_1$", xmin = -1.1, xmax = 1.5);
                yaxis("$\beta_2$", ymin = -1.1, ymax = 1.5);
            \end{asy}
            \caption{\(\lambda_1 > \lambda_2 > 0\)}
        \end{subfigure}
        \begin{subfigure}[b]{.3\linewidth}
            \begin{asy}
                import graph;
                import contour;
                
                size(80, 80);
                
                path p = (-1,0)--(-1,1)--(1,1)--(1,-1)--(-1,-1)--(-1,0);
                draw(p);
                xaxis("$\beta_1$", xmin = -1.1, xmax = 1.5);
                yaxis("$\beta_2$", ymin = -1.1, ymax = 1.5);
            \end{asy}
        \caption{\(\lambda_1 > \lambda_2 = 0\)}
    \end{subfigure}
    %\caption{Shapes of the sorted \(\ell_1\) norm}
    \end{figure}
\end{frame}

\begin{frame}{Equivalent formulations}

We've so far defined SLOPE as the \alert{constrained} optimization problem
\[
\begin{aligned}
    &\text{minimize}   && g(\beta; X) \\
    &\text{subject to} && \sum_{i=1}^p \lambda_i \lvert \beta \rvert_{(i)} \leq t,
\end{aligned}
\]
but, with a bit of notational abuse\footnote{Redefining \(\lambda\).}, this is equivalent to the \alert{unconstrained} problem
\[
\text{minimize} \quad g(\beta;\lambda) + J(\beta;\lambda),
\]
with \(J(\beta;\lambda) = \sum_{i=1}^p \lambda_i|\beta|_{(i)}\).
\end{frame}

\begin{frame}{Clustering Property}
    The sorted \(\ell_1\) norm induces clustering: setting absolute values of coefficients
    to same value\medskip
    
    Consider the two-dimensional case \(y = x_1\beta_1 + x_2\beta_2 + \varepsilon\).
    \begin{table}[htb]
        \centering
        
        \begin{tabular}{cc}
            \toprule
             lasso & SLOPE \\
             \midrule
             \(\lambda|\beta_1| + \lambda|\beta_2|\) & \(\lambda_1 |\beta|_{(1)} + \lambda_2 |\beta|_{(2)}\)\\
             \bottomrule
        \end{tabular}
    \end{table}
    and assume \(x_1\) and \(x_2\) are perfectly correlated, then
    \begin{itemize}
        \item the lasso will force one of the coefficients to zero, whilst
        \item SLOPE (provided \(\lambda_1 > \lambda_2 \geq 0\)) will set them to the same value.
    \end{itemize}
\end{frame}

\section{Selection of Regularization Sequence}

\begin{frame}[fragile]{Choice of \(\lambda\)}
\begin{itemize}
    \item \(\lambda\) is \(p\)-dimensional, which means there is \alert{considerable} 
          freedom in choosing it
    \item to make this problem manageable, we therefore assume that each \(\lambda_i\)
          is a function of \(p\), \(n\), \(i\), and some parameters
\end{itemize}
\end{frame}

\begin{frame}{Choice of \(\lambda\): BH}

\begin{columns}[T]
\begin{column}{0.45\linewidth}
    Inspiration for SLOPE comes from the desire to
    control of false discovery rate (FDR) in a regression setting, i.e.
    test the hypotheses
    \[
        H_{0,j}: \beta_j = 0 \qquad H_{1,j}: \beta_j \neq 0.
    \]
    
    \begin{block}{Benjamini--Hochberg (BH) procedure}
        Sort \(\hat\beta\) in non-decreasing order according to its absolute values and reject all hypothesis \(H_{(i)}\) for which \(i \leq i_{\text{BH}}\), where
        \[
            i_\text{BH} = \max\left\{i \mid |\hat\beta|_{(i)} \geq \sigma\Phi^{-1}\left(1 - \frac{iq}{2p}\right)\right\},
        \]
        where \(\Phi^{-1}\) is the probit function.
    \end{block}
\end{column}
\begin{column}{0.45\linewidth}
    \begin{figure}
        \centering
        \includegraphics[width=\linewidth]{figures/bh.pdf}
        \caption{BH correction.}
    \end{figure}
\end{column}
\end{columns}
\end{frame}

\begin{frame}{BH \(\lambda\) method for SLOPE}

The BH method for choosing the \(\lambda\) sequence in SLOPE sets
\[\lambda_i = \sigma\Phi^{-1}\left(1 - \frac{qi}{2p}\right).\]
Very similar to procedure from last slide.\medskip

It turns out that this sequence promises FDR control in \alert{orthogonal} settings \autocite[Theorem 1.1]{bogdan2015}, namely
\[
    \text{FDR}  \leq \frac{qp_0}{p},
\]
where \(p_0\) is the number of true null hypotheses. In other words, \(q\) sets an upper bound on FDR.
\end{frame}

\begin{frame}{FDR, Power, and Prediction Error}

Results from debiased\footnote{Select support using SLOPE or lasso and estimate coefficients using standard OLS.} SLOPE and lasso and cross-validated lasso show that SLOPE controls FDR as promised and has better predictive performance.

\begin{figure}
    \centering
    \includegraphics[width=\linewidth]{figures/bogdan-fig2.png}
    \caption{FDR, power, and mean-squared error (MSE) for lasso SLOPE using BH sequence. Predictors are i.i.d. generated from a normal distribution.}
\end{figure}
\end{frame}

\begin{frame}{FDR in Gaussian design}

When design is \alert{not} orthogonal, SLOPE with the BH sequence loses control of FDR
as the number of relevant predictors increase.

\begin{figure}
    \centering
    \includegraphics[width=0.5\linewidth]{figures/bogdan-fig5.png}
    \caption{FDR control for SLOPE with BH sequence for Gaussian i.i.d. design. \(p = 2n = 10000\).}
\end{figure}
\end{frame}

\begin{frame}{Choice of \(\lambda\): Gaussian Sequence}

In light of the problem with FDR control for non-orthogonal settings, \textcite{bogdan2015} consider also the \alert{Gaussian} sequence that sets
\[\lambda_i = \min\left(\lambda_{i-1}, \lambda^{\mathrm{BH}}_i\sqrt{1 + \frac{1}{n-i} \sum_{j < i}\lambda_i^2}\right),\]

\begin{columns}[T]
\begin{column}{0.35\linewidth}
which flattens out the sequence based on $n/p$ fraction.\medskip

Note, however, that for \(p \gg n\), this sequence reduces to the lasso.
\end{column}
\begin{column}{0.55\linewidth}
    \begin{figure}
    \centering
    \begin{tikzpicture}[/tikz/background rectangle/.style={fill={rgb,1:red,1.0;green,1.0;blue,1.0}, draw opacity={1.0}}, show background rectangle]
\begin{axis}[title={}, title style={at={{(0.5,1)}}, font={{\fontsize{14 pt}{18.2 pt}\selectfont}}, color={rgb,1:red,0.0;green,0.0;blue,0.0}, draw opacity={1.0}, rotate={0.0}}, legend style={color={rgb,1:red,0.0;green,0.0;blue,0.0}, draw opacity={1.0}, line width={1}, solid, fill={rgb,1:red,1.0;green,1.0;blue,1.0}, fill opacity={1.0}, text opacity={1.0}, font={{\fontsize{8 pt}{10.4 pt}\selectfont}}, at={(1.02, 1)}, anchor={north west}}, axis background/.style={fill={rgb,1:red,1.0;green,1.0;blue,1.0}, opacity={1.0}}, anchor={north west}, xshift={1.0mm}, yshift={-1.0mm}, width={145.4mm}, height={99.6mm}, scaled x ticks={false}, xlabel={}, x tick style={color={rgb,1:red,0.0;green,0.0;blue,0.0}, opacity={1.0}}, x tick label style={color={rgb,1:red,0.0;green,0.0;blue,0.0}, opacity={1.0}, rotate={0}}, xlabel style={, font={{\fontsize{11 pt}{14.3 pt}\selectfont}}, color={rgb,1:red,0.0;green,0.0;blue,0.0}, draw opacity={1.0}, rotate={0.0}}, xmajorgrids={true}, xmin={0.73}, xmax={10.27}, xtick={{2.0,4.0,6.0,8.0,10.0}}, xticklabels={{$2$,$4$,$6$,$8$,$10$}}, xtick align={inside}, xticklabel style={font={{\fontsize{8 pt}{10.4 pt}\selectfont}}, color={rgb,1:red,0.0;green,0.0;blue,0.0}, draw opacity={1.0}, rotate={0.0}}, x grid style={color={rgb,1:red,0.0;green,0.0;blue,0.0}, draw opacity={0.1}, line width={0.5}, solid}, axis x line*={left}, x axis line style={color={rgb,1:red,0.0;green,0.0;blue,0.0}, draw opacity={1.0}, line width={1}, solid}, scaled y ticks={false}, ylabel={}, y tick style={color={rgb,1:red,0.0;green,0.0;blue,0.0}, opacity={1.0}}, y tick label style={color={rgb,1:red,0.0;green,0.0;blue,0.0}, opacity={1.0}, rotate={0}}, ylabel style={, font={{\fontsize{11 pt}{14.3 pt}\selectfont}}, color={rgb,1:red,0.0;green,0.0;blue,0.0}, draw opacity={1.0}, rotate={0.0}}, ymajorgrids={true}, ymin={0.1294327982419891}, ymax={1.0051406129937785}, ytick={{0.2,0.4,0.6000000000000001,0.8,1.0}}, yticklabels={{$0.2$,$0.4$,$0.6$,$0.8$,$1.0$}}, ytick align={inside}, yticklabel style={font={{\fontsize{8 pt}{10.4 pt}\selectfont}}, color={rgb,1:red,0.0;green,0.0;blue,0.0}, draw opacity={1.0}, rotate={0.0}}, y grid style={color={rgb,1:red,0.0;green,0.0;blue,0.0}, draw opacity={0.1}, line width={0.5}, solid}, axis y line*={left}, y axis line style={color={rgb,1:red,0.0;green,0.0;blue,0.0}, draw opacity={1.0}, line width={1}, solid}, colorbar style={title={}}, point meta max={nan}, point meta min={nan}]
    \addplot[color={rgb,1:red,0.0;green,0.6056;blue,0.9787}, name path={7d13372e-d22a-4c66-ba80-a3440ec835d2}, draw opacity={1.0}, line width={1}, solid]
        coordinates {
            (1,0.353624447172169)
            (2,0.5876047525450598)
            (3,0.2533493190089544)
            (4,0.167488835407547)
            (5,0.41249537879137166)
            (6,0.9803564295574072)
            (7,0.2549733495259152)
            (8,0.7345424251056527)
            (9,0.4448311500326565)
            (10,0.1542169816783605)
        }
        ;
    \addlegendentry {y1}
\end{axis}
\end{tikzpicture}

\end{figure}
\end{column}
\end{columns}

\end{frame}

\begin{frame}{FDR Control with Gaussian Sequence}
    Gaussian sequence does a better job of controlling FDR in the Gaussian (but i.i.d.) case.
    
    \begin{figure}
        \centering
        \includegraphics[width=0.5\linewidth]{figures/bogdan-fig7.png}
        \caption{FDR control for SLOPE with Gaussian sequence for Gaussian i.i.d. design. \(p = 2n = 10000\).}
    \end{figure}
\end{frame}

\section{Parameter Selection for Regularization Sequence}

\begin{frame}[fragile]{Selecting parameters for \(\lambda\) sequence}
We've reduced the problem of specifying the entire \(\lambda\) sequence to specifying parameters \(\sigma\) and \(q\) but \alert{how do we choose these?}\medskip

\begin{block}{three options}
\begin{enumerate}
    \item we know \(\sigma\) and can assume that predictors are not very correlated
    \item use \autocite[Algorithm 5]{bogdan2015} and estimate \(\sigma\) using SLOPE to select support and OLS estimates to approximate \(\sigma\)
    \item cross-validation
\end{enumerate}
Choice depends on situation. \textbf{1} is of course usually intractable, \textbf{2} works well with low correlation and \(n > p\), \textbf{3} is attractive for prediction properties (but we lose FDR control).
\end{block}
\end{frame}

\begin{frame}{Cross-validation and the regularization path}
    \begin{columns}[c]
        \begin{column}{0.45\linewidth}
        If we are interested in cross-validation to select \(\sigma\) and \(q\), we will generally want to construct a \alert{regularization path} of \(\lambda\) sequences, \[\lambda^{(1)}, \lambda^{(2)}, \dots, \lambda^{(m)}.\]
        
        In addition, we want \(\lambda^{(1)}\) and \(\lambda^{(m)}\) to yield the \alert{intercept-only} model and \alert{almost-saturated} model respectively.\medskip
        
        \begin{block}{tuning parameters}
        \(\sigma\) (scale), \(q\) (shape)
        \end{block}
        \end{column}
        \begin{column}{0.45\linewidth}
            \begin{figure}
                \centering
                \includegraphics[width=\linewidth]{figures/slope-path.pdf}
                %\caption{Caption}
            \end{figure}
        \end{column}
    \end{columns}
\end{frame}

\section{SLOPE and lasso comparisons}

\begin{frame}{SLOPE and lasso regularization paths}
    \begin{figure}
        \centering
        \includegraphics[width=\linewidth]{figures/lassoslopepath.pdf}
        \caption{Comparison of SLOPE and lasso paths for correlated and uncorrelated data.}
    \end{figure}
\end{frame}

\begin{frame}{Grouping effects}
\begin{columns}
\begin{column}{0.45\linewidth}
    Differences between lasso and SLOPE are clear in block-correlated covariance matrices.
    
    \vspace{5ex}
    
    \begin{figure}
        \centering
        \includegraphics[width=0.8\linewidth]{figures/grouping-effect-sigma.pdf}
        \caption{Correlation structure.}
    \end{figure}
\end{column}
\begin{column}{0.45\linewidth}
    \begin{figure}
        \centering
        \includegraphics[width=0.9\linewidth]{figures/grouping-effect.pdf}
        \caption{Regularization paths.}
    \end{figure}
\end{column}
\end{columns}
\end{frame}

\begin{frame}{Performance and screening rules}

Sparsity-enforcing methods, such as lasso and SLOPE, can be \alert{very efficient}, particularly when \(p \gg n\) due to \alert{screening rules}\medskip

Screening rules are based on the idea that many solutions along the path will be sparse and that we can estimate this and \alert{discard} predictors before fitting the model.\medskip

We have developed a screening rule for SLOPE~\autocite{larsson2020b}

\end{frame}

\section{SLOPE extensions, implementation, and discussion}

\begin{frame}{Extensions of SLOPE}

Several extensions of SLOPE exists, all analagous to popular
lasso derivatives.

\begin{block}{Adaptive Bayesian SLOPE (ABSLOPE)}
    Semi-Bayesian approach that adapatively reweights penalties. See \textcite{jiang2019}.
\end{block}

\begin{block}{Group SLOPE}
    Sorted \(\ell_1\) norm regularization on group level
    \(\ell_2\) norm regularization on individual level. See \textcite{brzyski2018}.
\end{block}
\end{frame}

\begin{frame}{Software}

Most software is still under development, not quite mature.\medskip

\begin{block}{implementations}
\begin{itemize}
    \item SLOPE: \url{https://CRAN.R-project.org/package=SLOPE}
    \item Group SLOPE: \url{https://CRAN.R-project.org/package=grpSLOPE}
    \item ABSLOPE: \url{https://github.com/wjiang94/ABSLOPE}
\end{itemize}
\end{block}

\end{frame}

\begin{frame}{Topics for discussion}

\begin{itemize}
    \item How interested are you in 
    \begin{itemize}
        \item FDR control?
        \item model selection properties?
        \item prediction properties?
    \end{itemize}
    \item SLOPE penalizes stronger signals more than weaker ones---is this reasonable?
    \item choice of \(\lambda\) sequence has not been thoroughly investigated---any ideas? 
\end{itemize}
    
\end{frame}

\begin{frame}[allowframebreaks]{References}
  \bibliographytrue
  \printbibliography[heading=none]
\end{frame}

\end{document}
